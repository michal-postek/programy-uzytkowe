\documentclass[a4paper]{article}
\usepackage[left=2.5cm, right=2.5cm, top=2.5cm, bottom=2.5cm]{geometry}
\usepackage[utf8]{inputenc}
\usepackage[MeX]{polski}
\usepackage{caption}
\usepackage{multirow}

\title{Przedstawianie danych w LaTeX przy użyciu tabel}
\author{Michał Postek}
\date{6 Listopad 2022}
\begin{document}

\maketitle

\section{Strona tytułowa}

\subsection{Informacje o autorze}
\paragraph{Imię i nazwisko: Michał Postek}
\paragraph{I$^{\circ}$ Informatyka NS, grupa 4}

\subsection{Informacje o ćwiczeniach}
\paragraph{Tabele w LaTeX}
\paragraph{Programy użytkowe}
\paragraph{06.11.2022}

\newpage
\section{Zadanie 1}

\begin{table}[h]
\centering\caption{System decyzyjny modelujący problem diagnozy medycznej}
\begin{tabular}{c | c c c}
	\hline
	\hline
	Pacjent & Ból brzucha & Temperatura ciała & Operacja \\
	\hline
	u1 & Mocny & Wysoka & Tak \\
	u2 & Średni & Wysoka & Tak \\
	u3 & Mocny & Średnia & Tak \\
	u4 & Mocny & Niska & Tak \\
	u5 & Średni & Średnia & Tak \\
	u6 & Średni & Średnia & Nie \\
	u7 & Mały & Wysoka & Nie \\
	u8 & Mały & Niska & Nie \\
	u9 & Mocny & Niska & Nie \\
	u10 & Mały & Średnia & Nie \\
	\hline
	\hline
\end{tabular}
\end{table}

\newpage
\section{Zadanie 2}

\begin{table}[h]
\centering\caption{Działanie bramek logicznych}
\begin{tabular}{|c|c|c|c|c|c|c|c|c|}
	\hline
	\multicolumn{2}{|c|}{Argumenty} & \multicolumn{7}{|c|}{Bramki logiczne} \\
	\hline
	p & q & NOT p & NOT q & p AND q & p NAND q & p OR q & p NOR q & p XOR q \\
	\hline
	0 & 0 & 1 & 1 & 0 & 1 & 0 & 1 & 0 \\
	\hline
	0 & 1 & 1 & 0 & 0 & 1 & 1 & 0 & 1 \\
	\hline
	1 & 0 & 0 & 1 & 0 & 1 & 1 & 0 & 1 \\
	\hline
	1 & 1 & 0 & 0 & 1 & 0 & 1 & 0 & 0 \\
	\hline
\end{tabular}
\end{table}

\end{document}